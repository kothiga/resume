%%%%%%%%%%%%%%%%%%%%%%%%%%%%%%%%%%%%%%%%%%%%%%%%%%%%%%%%%%%%%%%%%%%%%%%%%%%%%%%%%%%%%%%%%%%%%%%%%%%%
%
% Austin Kothig Resume
%
% Based off the following Overleaf template
%   https://www.overleaf.com/latex/templates/recreating-business-insiders-cv-of-marissa-mayer/gtqfpbwncfvp
%
%%%%%%%%%%%%%%%%%%%%%%%%%%%%%%%%%%%%%%%%%%%%%%%%%%%%%%%%%%%%%%%%%%%%%%%%%%%%%%%%%%%%%%%%%%%%%%%%%%%%

\documentclass[10pt,a4paper,normalphoto]{altacv}

% Change the page layout if you need to
\geometry{left=1.0cm,right=1.0cm,top=1.0cm,bottom=1.0cm,columnsep=0.55cm}

% The paracol package lets you typeset columns of text in parallel
\usepackage{paracol}

% Change the font if you want to, depending on whether
% you're using pdflatex or xelatex/lualatex
% WHEN COMPILING WITH XELATEX PLEASE USE
% xelatex -shell-escape -output-driver="xdvipdfmx -z 0" mmayer.tex
\ifxetexorluatex
  % If using xelatex or lualatex:
  \setmainfont{Lato}
\else
  % If using pdflatex:
  \usepackage[default]{lato}
\fi

% Change the colors if you want to
\definecolor{VividPurple}{HTML}{3E0097}
\definecolor{SlateGrey}{HTML}{2E2E2E}
\definecolor{LightGrey}{HTML}{666666}
% \colorlet{name}{black}
% \colorlet{tagline}{PastelRed}
\colorlet{heading}{VividPurple}
\colorlet{headingrule}{VividPurple}
% \colorlet{subheading}{PastelRed}
\colorlet{accent}{VividPurple}
\colorlet{emphasis}{SlateGrey}
\colorlet{body}{LightGrey}

\hypersetup{
    colorlinks,
    linkcolor = VividPurple,
    urlcolor  = VividPurple,
    citecolor = VividPurple,
    anchorcolor = VividPurple
}

% Change some fonts, if necessary
% \renewcommand{\namefont}{\Huge\rmfamily\bfseries}
% \renewcommand{\personalinfofont}{\footnotesize}
% \renewcommand{\cvsectionfont}{\LARGE\rmfamily\bfseries}
% \renewcommand{\cvsubsectionfont}{\large\bfseries}

% Change the bullets for itemize and rating marker
% for \cvskill if you want to
\renewcommand{\cvItemMarker}{{\small\faAngleRight}}
\renewcommand{\cvRatingMarker}{\faCircle}
% ...and the markers for the date/location for \cvevent
% \renewcommand{\cvDateMarker}{\faCalendar*[regular]}
% \renewcommand{\cvLocationMarker}{\faMapMarker*}

\begin{document}
\name{Austin Kothig}
\tagline{MASc in Systems Design Engineering, BSc in Computer Science}
\photoR{2.75cm}{qr-code}

\personalinfo{%
    \email{\href{mailto:austin.kothig@gmail.com}{austin.kothig@gmail.com}}
    \homepage{\href{https://austin-kothig.ca}{austin-kothig.ca}}
    \github{\href{https://github.com/kothiga}{kothiga}}
    \\
    \linkedin{\href{https://www.linkedin.com/in/austin-kothig}{LinkedIn}}
    \scholar{\href{https://scholar.google.ca/citations?user=3oIJ5pUAAAAJ&hl=en}{Google Scholar}}
    \location{\href{https://maps.app.goo.gl/e4rprgPyH9jdUyf79}{Toronto, ON}}
}
\makecvheader

%% Depending on your tastes, you may want to make fonts of itemize environments slightly smaller
\AtBeginEnvironment{itemize}{\small}
\setitemize{label=\faAngleRight}
% leftmargin=15pt,noitemsep,topsep=0pt,parsep=0pt,partopsep=0pt,labelsep=0

%% Set the left/right column width ratio to 6:4.
\columnratio{0.6}

\vspace{-10px}
% Start a 2-column paracol. Both the left and right columns will automatically
% break across pages if things get too long.
\begin{paracol}{2}


%%%%%%%%%%%%%%%%%%%%%%%%%%%%%%%%%%%%%%%%%%%%%%%%%%%%%%%%%%%%%%%%%%%%%%%%%%%%%%%%%%%%%%%%%%%%%%%%%%%%
\cvsection{\faChartPie~~Experience}
%%%%%%%%%%%%%%%%%%%%%%%%%%%%%%%%%%%%%%%%%%%%%%%%%%%%%%%%%%%%%%%%%%%%%%%%%%%%%%%%%%%%%%%%%%%%%%%%%%%%

\cvevent{Software Developer}{IBM}{2023 -- 2025}{Toronto, ON}
\begin{itemize}
    \item I worked on a team that developed \textbf{profiling}, \textbf{tracing} and \textbf{backend} components of the observability platform \href{https://www.ibm.com/products/instana}{Instana}.
    \item I have led explorations into the usage of kernel level tracing technologies on \textbf{AIX} and \textbf{Linux} operating systems, including \href{https://www.ibm.com/docs/en/aix/7.1.0?topic=facility-running-probevue}{ProbeVue} and \href{https://github.com/cilium/ebpf}{Cilium/eBPF}.
    \item I have worked on integrating open source projects such as the \href{https://github.com/open-telemetry/opentelemetry-ebpf-profiler}{OpenTelemetry eBPF profiler} and the JVM \href{https://github.com/async-profiler/async-profiler}{Async-profiler} into Instana to provide additional profiling capabilities.
    \item I have previously worked on tracing components for PHP and have developed an open source \href{https://opentelemetry.io/}{OpenTelemetry} exporter which allows compatibility between the OpenTelemetry PHP SDK and Instana.
    \item Previously, I have provided open source \href{https://github.com/ClickHouse/ClickHouse/pulls?q=is%3Apr+is%3Amerged+author%3Akothiga+}{contributions} to the OLAP database \href{https://clickhouse.com/}{ClickHouse} to fix \textbf{Big-Endian} serialization bugs which then provided support for the s390x (IBM-Z) architecture.
\end{itemize}

\divider

\cvevent{Software Developer}{BlackBerry}{2022 -- 2023}{Waterloo, ON}
\begin{itemize}
    \item I was on a team that developed various customer-facing \textbf{software samples} to demonstrate development with the \href{https://blackberry.qnx.com/en/products/automotive/blackberry-ivy}{Intelligent Vehicle Data Platform (IVY)} framework. One such was a POC demo called Occupancy Identity Synthetic-Sensor, a machine learning system to \textbf{identify} and \textbf{remember} individuals inside a vehicle using facial embeddings.
    \item I led a technical investigation into developing a set of high-quality samples to show how \textbf{hardware data} can be integrated into the IVY framework.
\end{itemize}

\divider

\cvevent{Graduate Research Assistant}{SIRRL}{2019 -- 2022}{Waterloo, ON}
\begin{itemize}
    \item I was a member of the \textbf{Social and Intelligent Robotics Research Lab} (SIRRL) at the University of Waterloo. With this group, I worked with the \href{https://icub-tech-iit.github.io/documentation/}{iCub humanoid robot} and the \href{https://robots.ieee.org/robots/qtrobot/}{QTrobot}.
    \item For my research project I developed a \textbf{conceptual framework} for providing physiological data to robotic systems for real-time adaptation. Due to the COVID-19 pandemic, human participant research was restricted, which led me to implement a simple proof of concept scenario involving a \textbf{physiologically-aware} \href{https://doi.org/10.1109/RO-MAN50785.2021.9515383}{robotic exercise coach}.
    \item In 2020 I received a \textbf{research grant} from Microsoft which provided funding to hire two coop students that I had the opportunity to mentor for a summer.
    \item I have had the privilege to \textbf{collaborated} with colleagues on research projects and coauthored work related to people's expectations with an \href{https://doi.org/10.1109/LRA.2021.3083465}{unreliable robot} and exploration of if a robot can divert you from \href{https://doi.org/10.1145/3623809.3623879}{social engineering attacks}.
\end{itemize}

\divider

\cvevent{Undergraduate Research Assistant}{Cognitive Robotics Lab}{2017 -- 2019}{Lethbridge, AB}
\begin{itemize}
    \item I was a member of the Cognitive Robotics Lab at the University of Lethbridge, where I developed a \textbf{biologically inspired} \href{https://doi.org/10.1109/ROSE.2019.8790411}{audio attention system} for the iCub.
    \item During this time I received funding through \textbf{NSERC}, as well as \textbf{Mitacs} which allowed me to spend a semester abroad in Genoa, Italy collaborating with international colleagues.
\end{itemize}


%%%%%%%%%%%%%%%%%%%%%%%%%%%%%%%%%%%%%%%%%%%%%%%%%%%%%%%%%%%%%%%%%%%%%%%%%%%%%%%%%%%%%%%%%%%%%%%%%%%%
\switchcolumn
%%%%%%%%%%%%%%%%%%%%%%%%%%%%%%%%%%%%%%%%%%%%%%%%%%%%%%%%%%%%%%%%%%%%%%%%%%%%%%%%%%%%%%%%%%%%%%%%%%%%

%%%%%%%%%%%%%%%%%%%%%%%%%%%%%%%%%%%%%%%%%%%%%%%%%%%%%%%%%%%%%%%%%%%%%%%%%%%%%%%%%%%%%%%%%%%%%%%%%%%%
\cvsection{\faLeaf~~Summary}
%%%%%%%%%%%%%%%%%%%%%%%%%%%%%%%%%%%%%%%%%%%%%%%%%%%%%%%%%%%%%%%%%%%%%%%%%%%%%%%%%%%%%%%%%%%%%%%%%%%%

\begin{quote}
    \small
    \justifying
    \setlength\parindent{0pt}
    I am a Canadian software developer with three years of industry experience. My expertise lies in low-level C++ development, and I possess a strong attention to detail in both design and implementation. I am passionate about creating efficient, robust software solutions and continuously expanding my technical skills.
\end{quote}


%%%%%%%%%%%%%%%%%%%%%%%%%%%%%%%%%%%%%%%%%%%%%%%%%%%%%%%%%%%%%%%%%%%%%%%%%%%%%%%%%%%%%%%%%%%%%%%%%%%%
\cvsection{\faLaptopCode~~Skills}
%%%%%%%%%%%%%%%%%%%%%%%%%%%%%%%%%%%%%%%%%%%%%%%%%%%%%%%%%%%%%%%%%%%%%%%%%%%%%%%%%%%%%%%%%%%%%%%%%%%%

\cvachievement{\faCogs}{Programming}{
    \small
    \justifying
    \setlength\parindent{0pt}
    I am proficient in \textbf{C++} and \textbf{Python}, but also have some experience working with \textbf{Kotlin}, \textbf{Java}, \textbf{Rust} and \textbf{PHP}.
    I conduct my development primarily in \textbf{Docker} (containerized) environments on both \textbf{Linux} and \textbf{Mac} systems.
}

\divider

\cvachievement{\faBrain}{Personal Management}{
    \small
    \justifying
    \setlength\parindent{0pt}
    I thrive in both \textbf{collaborative} team settings and \textbf{independent} project work.
    I excel in \textbf{communication}, simplifying complex topics for discussion with diverse audiences.
    My approach to problem-solving involves \textbf{identifying} key questions, \textbf{designing} effective solutions, and \textbf{iteratively} implementing them efficiently. 
    \textbf{}
}


%%%%%%%%%%%%%%%%%%%%%%%%%%%%%%%%%%%%%%%%%%%%%%%%%%%%%%%%%%%%%%%%%%%%%%%%%%%%%%%%%%%%%%%%%%%%%%%%%%%%
\cvsection{\faGraduationCap~~Education}
%%%%%%%%%%%%%%%%%%%%%%%%%%%%%%%%%%%%%%%%%%%%%%%%%%%%%%%%%%%%%%%%%%%%%%%%%%%%%%%%%%%%%%%%%%%%%%%%%%%%

\cveducation{MASc in Systems Design Engineering}{University of Waterloo}{}{}
\vspace{-5px}
\begin{itemize}
    \item Advisor: Prof. Kerstin Dautenhahn
    \item Thesis: \say{\textit{Accessible Integration of Physiological Adaptation in Human-Robot Interaction}} \href{https://uwspace.uwaterloo.ca/handle/10012/17462}{\faLink}
\end{itemize}

\vspace{1px}

\cveducation{BSc\ in Computer Science}{University of Lethbridge}{}{}
\vspace{-5px}
\begin{itemize}
    \item Graduated with Co-operative Education
    \item Exchange term at the Italian Institute of Tech. \href{https://www.ulethbridge.ca/unews/article/mitacs-award-winner-takes-studies-overseas-italy}{\faLink}
\end{itemize}


%%%%%%%%%%%%%%%%%%%%%%%%%%%%%%%%%%%%%%%%%%%%%%%%%%%%%%%%%%%%%%%%%%%%%%%%%%%%%%%%
\cvsection{\faTrophy~~Achievements}
%%%%%%%%%%%%%%%%%%%%%%%%%%%%%%%%%%%%%%%%%%%%%%%%%%%%%%%%%%%%%%%%%%%%%%%%%%%%%%%%

\cvachievement{\faLandmark}{Awards and Honors}{
    Select Academic Achievements
}
\begin{itemize} %\footnotesize
    \item (2020) Microsoft AI for Social Good Research Grant
    \item (2019) Engineering Dean's Entrance Award
    \item (2019) NSERC Undergraduate Student Research Award
    \item (2018) Mitacs Globalink Research Award
    \item (2018) Chinook Summer Research Award
    \item (2018) Simpson-Markinch Award
    \item (2017, 2018) Jason Lang Scholarship
\end{itemize}


% %%%%%%%%%%%%%%%%%%%%%%%%%%%%%%%%%%%%%%%%%%%%%%%%%%%%%%%%%%%%%%%%%%%%%%%%%%%%%%%%%%%%%%%%%%%%%%%%%%%%
% \cvsection{Publications}
% %%%%%%%%%%%%%%%%%%%%%%%%%%%%%%%%%%%%%%%%%%%%%%%%%%%%%%%%%%%%%%%%%%%%%%%%%%%%%%%%%%%%%%%%%%%%%%%%%%%%

% \begin{itemize} %\footnotesize
%     \item (2023) Pasquali D, \underline{Kothig A}, Aroyo AM, Mu{\~n}oz J, Dautenhahn K, Bencetti S, Francesco R, Sciutti A. \say{That's not a Good Idea: A robot changes your behavior against social engineering}. In International Conference on Human-Agent Interaction
%     \item (2021) \underline{Kothig A}, Mu{\~n}oz J, Akgun SA, Aroyo AM, Dautenhahn K. \say{Connecting humans and robots using physiological signals -- closing-the-loop in HRI}. In IEEE International Conference on Robot and Human Interactive Communication
%     \item (2021) Aroyo AM, Pasquali D, \underline{Kothig A}, Rea F, Sandini G, Sciutti A. \say{Expectations vs. Reality: Unreliability and transparency in a treasure hunt game with iCub}. IEEE Robotics and Automation Letters
%     \item (2020) \underline{Kothig A}, Mu{\~n}oz J, Mahdi H, Aroyo AM, Dautenhahn K. \say{HRI Physio Lib: A software framework to support the integration of physiological adaptation in HRI}. In International Conference on Social Robotics
%     \item (2020) Rea F, \underline{Kothig A}, Grasse L, Tata M. \say{Speech envelope dynamics for noise-robust auditory scene analysis in robotics}. International Journal of Humanoid Robotics
%     \item (2019) \underline{Kothig A}, Ilievski M, Grasse L, Rea F, Tata M. \say{A Bayesian system for noise-robust binaural sound localisation for humanoid robots}. In IEEE International Symposium on Robotic and Sensors Environments '13
% \end{itemize}


%%%%%%%%%%%%%%%%%%%%%%%%%%%%%%%%%%%%%%%%%%%%%%%%%%%%%%%%%%%%%%%%%%%%%%%%%%%%%%%%%%%%%%%%%%%%%%%%%%%%
\cvsection{\faTree~~Hobbies}
%%%%%%%%%%%%%%%%%%%%%%%%%%%%%%%%%%%%%%%%%%%%%%%%%%%%%%%%%%%%%%%%%%%%%%%%%%%%%%%%%%%%%%%%%%%%%%%%%%%%

\cvtag{\faMountain~~Rock Climbing}
\cvtag{\faRunning~~Distance Running}
\cvtag{\faCamera~~Urban Photography}
\cvtag{\faFlagCheckered~~Motorsport}

\end{paracol}
\end{document}
